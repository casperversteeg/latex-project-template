\documentclass[onehalfspace]{Report}

\title{Title}				% Title
\date{\today}				% Date
\newcommand{\theauthor}{Author Name}
\newcommand{\course}{Course Number} % Course number
\newcommand{\institute}{ Duke University } % Institute
\newcommand{\versionNumber}{Version number} % version number

% \addbibresource{./bibliography/References.bib}

\begin{document}
\begin{singlespace}
\documentclass[class=Report, crop=false]{standalone}
\makeatletter
\let\thetitle\@title
\makeatother
\begin{document}
\begin{titlepage}
    \centering
      \vspace*{0.5 cm}
      \includegraphics[height=2.5in, trim = {1350px 0 0 0}, clip]{../images/Color-Duke-University-logo.jpg}\\[1.0 cm]	% University Logo
    \rule{\linewidth}{0.2 mm} \\[0.4 cm]
    { \Huge \bfseries \thetitle }\\
    \rule{\linewidth}{0.2 mm} \\[1cm]
    \textsc{\LARGE \course}\\[0.3 cm]				% Course Code
    \textsc{\Large Duke University}\\[0.3 cm]	% University Name
    \textsc{\large\thedate}\\[1.0cm]
    \versionNumber \\[1.0cm]

  \vspace{0.5in}
  \noindent \centering \begin{tabular}{c}
  \makebox[2.5in]{\hrulefill}\\
  {\Large \theauthor}\\[0.1cm]
  {Mechanical Engineering}\\
  \end{tabular}
  \vspace{1in}
\end{titlepage}

\end{document}


\tableofcontents
\end{singlespace}

%%%%%%%%%%%%%%%%%%%%%%%%%%%%%%%%%%%%%%%%%%
\documentclass[class=Report, crop=false]{standalone}
\begin{document}
\section*{Revision Table}
\addcontentsline{toc}{section}{Revision Table}

\begin{table}[H]
\centering
\scriptsize
\tabulinesep = 2mm
\begin{tabu}{|P{3.5in}|P{1.5in}|P{1in}|} \hline
\textbf{Changes} & \textbf{Author(s)} & \textbf{Version} \\ \hline
[yyyy/mm/dd]---describe changes & last,first & release . major . minor\\ \hline
 & & \\ \hline
 & & \\ \hline
 & & \\ \hline
 & & \\ \hline
 & & \\ \hline
 & & \\ \hline
 & & \\ \hline
\end{tabu}
\end{table}
\end{document}


\input{sections/abstract}

\pagebreak
\documentclass[class=Report, crop=false]{standalone}
\begin{document}
\section{Introduction}

\lipsum[5-6]

\end{document}


\begin{definition}{lenghts}{len}
\lipsum[1]
\end{definition}

\begin{theorem}{Pythagorean Theorem}{pyth}
\label{pythagorean}
This is a theorem about right triangles and can be summarised in the next
equation
\[ x^2 + y^2 = z^2 \]
\tcblower
 To prove it by contradiction try and assume that the statement is false, proceed from there and at some point you will arrive to a contradiction. Hence \[ a^2 = b^2 + c^2. \]%
\end{theorem}

\begin{example}{}{}
\lipsum[1]
\end{example}

\begin{exercise}{}{}
  \lipsum[2]
\end{exercise}

%%%%%%%%%%%%%%%%%%%%%%%%%%%%%%%%%%%%%%%%%%
\pagebreak
\section*{Appendices}
\addcontentsline{toc}{section}{Appendices}

\begin{appendices}

\documentclass[class=Report, crop=false]{standalone}

\begin{document}
\section{Mathematical Preliminaries}

\lipsum[1-2]

\end{document}


\end{appendices}

% \addcontentsline{toc}{chapter}{Bibliography} %adds Bibliography to your table of contents
% \printbibliography

\end{document}
