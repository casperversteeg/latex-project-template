\documentclass[singlespace, lite]{TechnicalVolume}

\title{Title}				% Title
\date{\today}				% Date
\newcommand{\CMtitle}{CM-title} % CM title

\addbibresource{./bibliography/References.bib}

\begin{document}
\begin{singlespace}
\documentclass[class=Report, crop=false]{standalone}
\makeatletter
\let\thetitle\@title
\makeatother

\begin{document}

  \vspace*{0.5cm}
  \begin{center} \Huge \bfseries \thetitle \end{center}
  \vspace*{1.0cm}
  
  \noindent\small {\bfseries SBIR Rights Notice}

  \noindent These SBIR data are furnished with SBIR rights under Contract No.80NSSC23CA182, 
  date of award September 12, 2023. For a period of 20 years, starting from the date of award, 
  the Government will use these data for Government purposes only, and they shall not be 
  disclosed outside the Government (including disclosure for procurement purposes) during 
  such period without the permission of the Contractor, (unless specifically permitted 
  elsewhere in the contract) except that, subject to the foregoing use and disclosure 
  prohibitions, these data may be disclosed for use by support Contractors. After the SBIR 
  protection period ends, the Government has a paidup license to use, and to authorize others 
  to use on its behalf, these data for Government purposes, but is relieved of all disclosure 
  prohibitions and assumes no liability for unauthorized use of these data by third parties. 
  This notice shall be affixed to any reproductions of these data, in whole or in part.

  \vspace*{0.5cm}

\end{document}


\tableofcontents
\end{singlespace}


%%%%%%%%%%%%%%%%%%%%%%%%%%%%%%%%%%%%%%%%%%
\pagebreak
\documentclass[class=Report, crop=false]{standalone}

\begin{document}
\section*{Abstract}\addcontentsline{toc}{section}{Abstract}

\lipsum[1-2]

\end{document}


\documentclass[class=Report, crop=false]{standalone}
\begin{document}
\section{Introduction}

\lipsum[5-6]

\end{document}


\begin{definition}{lenghts}{len}
\lipsum[1]
\end{definition}

\begin{theorem}{Pythagorean Theorem}{pyth}
\label{pythagorean}
This is a theorem about right triangles and can be summarised in the next
equation
\[ x^2 + y^2 = z^2 \]
\tcblower
 To prove it by contradiction try and assume that the statement is false, proceed from there and at some point you will arrive to a contradiction. Hence \[ a^2 = b^2 + c^2. \]%
\end{theorem}

\begin{example}{}{}
\lipsum[1]
\end{example}

\begin{exercise}{}{}
  \lipsum[2]
\end{exercise}

%%%%%%%%%%%%%%%%%%%%%%%%%%%%%%%%%%%%%%%%%%
\pagebreak
\section*{Appendices}
\addcontentsline{toc}{section}{Appendices}

\begin{appendices}

\documentclass[class=Report, crop=false]{standalone}

\begin{document}
\section{Mathematical Preliminaries}

\lipsum[1-2]

\end{document}


\end{appendices}

\addcontentsline{toc}{section}{Bibliography} %adds Bibliography to your table of contents
\nocite{*}
\printbibliography[title={Bibliography}]

\end{document}
