\documentclass[singlespace]{Presentation}

% \addbibresource{./bibliography/References.bib}

\title{Title}				% Title
\subtitle{}         % Subtitle
\date{\today}				% Date
\titlegraphic{\dukelogo{1in}} % Title graphic
\author{Casper Versteeg}
\institute[Duke University]{Pratt School of Engineering \\Duke University }
\newcommand{\course}{Course Number} % Course number
\newcommand{\versionNumber}{Version number} % version number

\begin{document}
\begin{frame}[noframenumbering]
  \titlepage
\end{frame}

\begin{frame}
  \frametitle{Overview}
  \tableofcontents
\end{frame}

\section{Section 1}
\begin{frame}
\frametitle{There Is No Largest Prime Number}
\framesubtitle{The proof uses \textit{reductio ad absurdum}.}
\begin{theorem*}{}{}
There is no largest prime number.
\end{theorem*}
\framebreak
\begin{enumerate}
  \item<1-| alert@1> Suppose $p$ were the largest prime number.
  \item<2-> Let $q$ be the product of the first $p$ numbers.
  \item<3-> Then $q+1$ is not divisible by any of them.
  \item<1-> But $q + 1$ is greater than $1$, thus divisible by some prime number not in the first $p$ numbers.
\end{enumerate}
\end{frame}

\begin{frame}{A longer title}
  \begin{itemize}
    \item one\\
    \item two\\
  \end{itemize}
\end{frame}

\section{Section Name}
\subsection{subsec}
\begin{frame}{Slide Name 1}
\end{frame}

\begin{frame}{Slide Name 2}
\end{frame}
%
\section{References}
\begin{frame}[allowframebreaks]{References}
  % \nocite{*}
  % \printbibliography
\end{frame}

\end{document}
